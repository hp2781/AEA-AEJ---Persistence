\begin{itemize}
\item Do not use an "Introduction" heading. Begin your introductory material
before the first section heading.
\item Avoid style markup (except sparingly for emphasis).
\item Avoid using explicit vertical or horizontal space.
\item Captions are short and go below figures but above tables.
\item The tablenotes or figurenotes environments may be used below tables
or figures, respectively, as demonstrated below.
\item If you have difficulties with the math time package, adjust the package
options appropriately for your platform. If you can't get it to work, just
remove the package or see our technical support document online (please
refer to the author instructions).
\item If you are using an appendix, it goes last, after the bibliography.
Use regular section headings to make the appendix headings.
\item If you are not using an appendix, you may delete the appendix command
and sample appendix section heading.
\item Either the natbib package or the harvard package may be used with bibtex.
To include one of these packages, uncomment the appropriate usepackage command
above. Note: you can't use both packages at once or compile-time errors will result.
\end{itemize}




\begin{figure}
Figure here.

\caption{Caption for figure below.}
\begin{figurenotes}
Figure notes without optional leading.
\end{figurenotes}
\begin{figurenotes}[Source]
Figure notes with optional leadin (Source, in this case).
\end{figurenotes}
\end{figure}

Sample table:

\begin{table}
\caption{Caption for table above.}

\begin{tabular}{lll}
& Heading 1 & Heading 2 \\ 
Row 1 & 1 & 2 \\ 
Row 2 & 3 & 4%
\end{tabular}
\begin{tablenotes}
Table notes environment without optional leadin.
\end{tablenotes}
\begin{tablenotes}[Source]
Table notes environment with optional leadin (Source, in this case).
\end{tablenotes}
\end{table}




\begin{comment}



Sample citation of Nash paper \cite{Nash}.

References here (manual or bibTeX). If you are using bibTeX, add your bib file 
name in place of BibFile in the bibliography command.
% Remove or comment out the next two lines if you are not using bibtex.

\end{comment}



CoreLogic Monthly Chart Pack

Released August 2018

Good morning everyone

Please find a link below to a copy of this month’s indices chart pack which covers off on housing market data to the end of July 2018.  Please feel free to forward a copy of this pack to your customers as you see fit and remember that the CoreLogic data contained within is available for purchase. 

 Monthly highlights

·       National dwelling values fell for the 10th consecutive month in July 2018, down 
-0.6% which was the greatest monthly fall in values since September 2011.  Over the month, combined capital city values fell by -0.6% while the combined regional markets recorded a -0.4% fall.

·       Values rose over the month in Brisbane, Darwin and Canberra, were unchanged in Hobart and they fell elsewhere.

·       Over the three months to July 2018, dwelling values nationally were -0.9% lower which was their largest fall since January 2012 with the combined capital cities recording a fall of -1.1% while the combined regional markets recorded an fall of -0.2%.

·       Values rose over the past three months in Brisbane, Adelaide and Perth but fell elsewhere.

·       Dwelling values fell by -1.6% nationally over the past year which was their largest annual decline since August 2012.

·       Despite the slowing conditions all capital cities except for Sydney, Melbourne Perth and Darwin have recorded value rises over the year however, the decline in Sydney values is the largest since march 2009 and the fall in Melbourne is the greatest since November 2012.   

·       The combined regional markets have recorded growth in values over the past year (1.6%) while the combined capital cities have recorded a decline of (-2.4%) although both regions are seeing a slowing trend in value changes.

·       Transaction volumes are much lower over the year down -9.8% nationally with Adelaide the only capital city in which sales volumes were higher over the year.

·       Rental rates fell by -0.2% over the month to be -0.1% lower over the past three months and 1.6% higher over the past year. 

·       Rental rates rose over the year across all capital cities except for Sydney and Darwin with Sydney recording its largest annual fall in rents on record.

·       Rental yields have started to lift from their record lows as rental growth outpaces value growth, yields are currently recorded at 3.72% up from 3.63% in July 2017.

·       Perth and Hobart are the only capital cities in which gross rental yields are now lower than they were a year ago.

·       The length of time it takes to sell a property has increased relative to a year ago in all capital cities except for Adelaide, Hobart and Canberra where properties are selling quicker.

·       The volume of new and total stock advertised for sale nationally is lower than a year ago.  Across the cities, total listings are much higher than they were a year ago Sydney and Melbourne and marginally higher in Brisbane and Perth while they are lower elsewhere.

·       Population growth remains strong however, an increasing number of residents are leaving NSW with interstate migration to Qld accelerating.

·       Dwelling approvals increased by 1.6% in June 2018, with increases for houses and units.  Although approvals are down from their peak they remain at elevated levels that are well above long-term averages.

·       In terms of housing finance, investor demand is waning with owner occupiers now the dominant source of demand.  In NSW and Vic, recent removals of stamp duty for first time buyers has resulted in a surge in demand from this sector.

·       The expansion of housing credit continued to slow in June 2018 with credit to investors expanding at its slowest annual pace on record.

·       Official interest rates remain at 1.5% with the market currently not expecting a full 25 basis point change in the cash rate over the next 18 months.




CoreLogic July Home Value Index results out today

 

EMBARGOED UNTIL 10AM ADET

 

Housing downturn gathers momentum in July, with national dwelling values falling at the fastest annual rate since 2012

 

Key Findings:

 

The CoreLogic July home value results out today confirmed that national dwelling values continued their weak run, with both capital city and regional dwelling values trending lower over the past three months.

 

·         National dwelling values slipped 0.6% over the month to be down 0.9% over the rolling quarter and 1.6% lower over the past twelve months; the largest annual fall since August 2012. 

·         Since peaking in September last year, the Australian housing market has recorded a cumulative 1.9% fall in value; a relatively mild downturn to date considering values remain 31% higher than they were five years ago. 

·         Weakness in dwelling values driven by the long running declines in Perth & Darwin along with an acceleration in the rate of decline across Sydney and Melbourne and slowing growth rates across most of the remaining regions.

·         Best performing capital city: Hobart +1.1%

·         Weakest performing capital city: Melbourne -1.8%

·         Highest rental yield: Darwin 5.7%

·         Lowest rental yields: Melbourne 3.0%



% text lines or sentences

\iffalse Should they invest at the high level now and push the prices upwards for other market participants or risk paying a higher price later themselves if other market participants start buying.\fi